\documentclass[12pt]{article}

\usepackage[utf8]{inputenc}      
\usepackage{graphicx}           
\usepackage{biblatex}            
\addbibresource{references.bib}  

\title{Jane Eyre: A Literary Analysis}
\author{Ramdani Lysa}
\date{16 January 2026}

\begin{document}

\maketitle

% Centered image with caption
\begin{center}
\includegraphics[width=0.3\textwidth]{jane_eyre_cover.jpg}
\end{center}

\section*{Introduction}

\noindent
\textit{Jane Eyre}, written by Charlotte Brontë \cite{bronte2002}, is a cornerstone of English literature that chronicles the life of an orphaned girl who grows into a resilient and independent woman despite societal constraints. The novel explores themes of love, morality, and social class, emphasizing the challenges faced by women in Victorian England. Through Jane’s journey, Brontë highlights the significance of personal integrity, emotional resilience, and self-respect. This analysis examines how Jane’s experiences reflect the delicate balance between love, independence, and societal expectations.

\section*{Development}

\noindent
In \textit{Jane Eyre} \cite{bronte2002}, Brontë portrays love as deeply intertwined with morality, personal values, and social limitations. Jane’s relationship with Mr. Rochester exemplifies the tension between personal desire and societal expectation, complicated by differences in wealth, status, and past secrets. Despite these challenges, Jane maintains her dignity and independence, demonstrating that authentic love should never compromise moral principles. Scholars have noted that emotional honesty and self-respect are essential for genuine relationships, and Jane’s struggles highlight the pressures faced by women navigating gender and class norms in Victorian society \cite{coste2005}. Ultimately, Brontë’s narrative shows that love and personal integrity can coexist, even within a rigid social hierarchy.

\section*{Conclusion}

\noindent
In conclusion, \textit{Jane Eyre} remains a timeless exploration of love, independence, and social expectations. Charlotte Brontë’s portrayal of Jane illustrates the strength derived from combining personal integrity with emotional courage, proving that societal pressures need not obstruct meaningful relationships. The novel continues to resonate because it challenges conventional norms while celebrating the enduring importance of self-respect, resilience, and moral fortitude in both personal and romantic life.

\printbibliography

\end{document}
